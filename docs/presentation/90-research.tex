\frame {
    \frametitle{\LARGE Эксперимент}
    \pgfplotsset{tick label style={font=\tiny\bfseries},
        label style={font=\small},
        legend style={font=\tiny}
    }
    \begin{columns}
        \begin{column}{0.6\textwidth}
            Кэширование данных показало снижение времени отклика системы вплоть до 39 раз, при условии нахождения запрашиваемых данных в кэше
            \begin{itemize}
                \item 23.6~c --- получение информации из базы данных;
                \item 0.6~с --- получение информации из кэша.
            \end{itemize}
        \end{column}
        \begin{column}{0.4\textwidth}
            \begin{tikzpicture}[scale=.5]
                \begin{axis}[
                    legend pos=north west,
                    ymajorgrids=true,
                    xtick={32,64,128,256},
                    xlabel={Количество слоев в нейронной сети (по 16 нейронов)},
                    ylabel={Время, с}]
                    \addplot table[x=size,y=db,col sep=comma] {inc/csv/res16.csv};
                    \addplot table[x=size,y=cache,col sep=comma] {inc/csv/res16.csv};
                    \legend{Без кэширования, С кэшированием}
                \end{axis}
                \end{tikzpicture}
                \vskip.5em
                \begin{tikzpicture}[scale=.5]
                    \begin{axis}[
                        legend pos=north west,
                        ymajorgrids=true,
                        xtick={32,64,128,256},
                        xlabel={Количество слоев в нейронной сети (по 128 нейронов)},
                        ylabel={Время, с}]
                        \addplot table[x=size,y=db,col sep=comma] {inc/csv/res128.csv};
                        \addplot table[x=size,y=cache,col sep=comma] {inc/csv/res128.csv};
                        \legend{Без кэширования, С кэшированием}
                    \end{axis}
                \end{tikzpicture}
        \end{column}
    \end{columns}
}
