\chapter{Технологическая часть}

В данном разделе обосновывается выбор конкретных СУБД для решения задачи, описывается общая архитектура реализуемого программного продукта. Кроме того, приводится описание архитектуры реализуемого приложения: описание архитектурного паттерна, используемого при проектировании, описание конкретных компонентов приложения, а так же - определяются средства реализации приложения, описывается интерфейс взаимодействия с приложением, приводятся детали реализации разрабатываемого приложения.

\section{Обзор СУБД}

В данном подразделе будут рассмотрены популярные СУБД, которые могут быть использованы для реализации хранения в разрабатываемом программном продукте.\\

\noindent\textbf{Oracle Database}\\

Oracle Database~\cite{oracle} -- объектно-реляционная система управления базами данных разрабатываемая компанией Oracle \cite{oracle-company}. На данный момент, рассматриваемая СУБД является наиболее популярной в мире. \cite{oracle-popular}

Все транзакции Oracle Database обладают свойствами ACID~\cite{acid}, поддерживает триггеры, внешние ключи и хранимые процедуры. Данная СУБД подходит для разнообразных рабочих нагрузок и может использоваться практически в любых задачах. Особенностью Oracle Database является быстрая работа с большими массивами данных.\\

\noindent\textbf{MySQL}\\

MySQL~\cite{mysql} -- свободная реляционная система управления базами данных. Разработку и поддержку MySQL осуществляет корпорация Oracle.

Рассматриваемая СУБД имеет два основных движка хранения данных: InnoDB~\cite{innodb} и myISAM~\cite{myisam}. Движок InnoDB полностью совместим с принципами ACID, в отличии от движка myISAM. СУБД MySQL подходит для использования при разработке веб-приложений, что объясняется очень тесной интеграцией с популярными языками PHP \cite{php} и Perl \cite{perl}.

\newpage

\noindent\textbf{PostgreSQL}\\

PostgreSQL~\cite{postgresql} -- это свободно распространяемая объектно-реляционная система управления базами данных, наиболее развитая из открытых СУБД в мире и являющаяся реальной альтернативой коммерческим базам данных \cite{postgresql-fact}.

PostgreSQL предоставляет транзакции, обладающие свойствами ACID, автоматически обновляемые представления, материализованные представления, триггеры, внешние ключи и хранимые процедуры. Данная СУБД предназначена для обработки ряда рабочих нагрузок, от отдельных компьютеров до хранилищ данных или веб-сервисов с множеством одновременных пользователей. 

\section{Обзор in-memory NoSQL СУБД}

\noindent\textbf{Redis}\\

Redis~\cite{redis} -- резидентная система управлениями базами данных класса NoSQL с открытым исходным кодом. Основной структурой данных, с которой работает Redis является структура типа <<ключ-значение>>. Данная СУБД используется как для хранения данных, так и для реализации кэшей и брокеров сообщений.

Redis хранит данные в оперативной памяти и снабжена механизмом <<снимков>> и журналирования, что обеспечивает постоянное хранение данных. Существует поддержка репликации данных типа master-slave, транзакций и пакетной обработки команд.

Все данные Redis хранит в виде словаря, в котором ключи связаны со своими значениями. Ключевое отличие Redis от других хранилищ данных заключается в том, что значения этих ключей не ограничиваются строками. Поддерживаются следующие абстрактные типы данных:

\begin{itemize}
	\item строки;
	\item списки;
	\item множества;
	\item хеш-таблицы;
	\item упорядоченные множества.
\end{itemize}

Тип данных значения определяет, какие операции доступные для него; поддерживаются высокоуровневые операции: например, объединение, разность или сортировка наборов.\\

\noindent\textbf{Tarantool}\\

Tarantool~\cite{tarantool} -- это платформа in-memory вычислений с гибкой схемой хранения данных для эффективного создания высоконагруженных приложений. Включает себя базу данных и сервер приложений на языке программирования Lua~\cite{lua}.

Tarantool обладает высокой скоростью работы по сравнению с традиционными СУБД. При этом, в рассматриваемой платформе для транзакций реализованы свойства ACID, репликация \texttt{master-slave}~\cite{master-slave} и \texttt{master-master}~\cite{master-master}, как и в традиционных СУБД.

Для хранения данных используется кортежи (англ. tuple) данных. Кортеж -- это массив не типизированных данных. Кортежи объединяются в спейсы (англ. space), аналоги таблицы из реляционной модели хранения данных.

В рассматриваемой СУБД реализованы два движка хранения данных: memtx~\cite{memtx-vinyl} и vinyl~\cite{memtx-vinyl}. Первый хранит все данные в оперативной памяти, а второй на диске. Для каждого спейса можно задавать различный движок хранения данных. 

Каждый спейс должен быть проиндексирован первичным ключом. Кроме того, поддерживается неограниченное количество вторичных ключей. Каждый из ключей может быть составным.

В Tarantool реализован механизм <<снимков>> текущего состояния хранилища и журналирования всех операций, что позволяет восстановить состояние базы данных после ее перезагрузки.

\section{Выбор СУБД для решения задачи}

Для решения задачи хранения конфигураций нейронный сетей была выбрана СУБД PostgreSQL, потому что данная СУБД имеет поддержку языка \texttt{plpython3u}~\cite{plpython3u}, который упрощает процесс интеграции базы данных в разрабатываемое приложение. Кроме того, PostgreSQL проста в развертывании.

Для кэширования данных была выбрана СУБД Tarantool, так как она проста в развертывании и переносимости, и имеет подходящие коннекторы для базы данных PostgreSQL~\cite{postgresql-fact}.

\section{Архитектура приложения}

Предполагается, что разрабатываемое приложения является частью некоторого большого сервиса. Доступ к данным, хранящимся в приложении предоставляется посредством \texttt{REST API}~\cite{rest}.

Серверная часть приложения взаимодействует с базами данных посредством коннекторов, позволяющих выполнять запросы к базе данных на языке программирования, используемом для разработки приложения.

Общая схема взаимодействия с базой данных представлена на Рисунке~\ref{img:arch}

\imgw{arch}{ht!}{0.9\textwidth}{Общая схема взаимодействия с базой данных}

В качестве основного паттерна проектирования архитектуры приложения будет использован \texttt{MVP}~\cite{mvp}. Разрабатываемое приложение будет состоять из следующих компонентов:

На Рисунке~\ref{img:components} представлены следующие компоненты:
\begin{itemize}
    \item \texttt{Main} --- основной компонент приложения, содержит точку входа, а так же производит конфигурирование других компонентов приложения;
    \item \texttt{Config Manager} --- компонент, предоставляющий данные для конфигурирования приложения;
    \item \texttt{Authorizer} --- компонент, реализующий механизмы аутентификации и авторизации в системе \cite{auth};
    \item \texttt{View} --- компонент, являющийся частью паттерна \texttt{MVP} - предоставляет доступ к системе. Так как в данной курсовой работе реализуется доступ посредством \texttt{REST API}, то данный компонент должен быть представлен обработчиками \texttt{HTTP}-запросов \cite{http};
    \item \texttt{Interactors} --- компонент, реализующий \texttt{Presenter} паттерна;
    \item \texttt{Entites} --- компонент, содержащий реализации доменных моделей --- \texttt{Model} паттерна;
    \item \texttt{Database} --- компонент, предоставляющий доступ к базе данных, хранящей конфигурации нейронных сетей;
    \item \texttt{Cache} --- компонент, предоставляющий доступ к кэшу приложения.
\end{itemize}

Соответствующая диаграмма компонентов разрабатываемого приложения приведена на Рисунке~\ref{img:components}.

\imgw{components}{ht!}{\textwidth}{Диаграмма компонентов}

\section{Средства реализации}

Разрабатываемое приложение будет являться web-сервером, предоставляющим \texttt{REST API}. В связя с этим в качестве языка разработки будет использоваться язык \texttt{Golang}~\cite{golang}, так как этот язык создан для разработки различных web-приложений и для него существуют готовые коннекторы к выбранным базам данных.

Для реализации \texttt{REST API} был использован фреймворк \texttt{Gin Gonic}~\cite{gingonic}.

Для коммуникации серверной части приложения с базами данных были использованы коннекторы \texttt{gorm}~\cite{gorm} для PostgreSQL и \texttt{go-tarantool}~\cite{go-tarantool} для tarantool соответственно.

Для упаковки приложения в готовый продукт была выбрана система контейнеризации \texttt{Docker}~\cite{docker} и система управления развертыванием контейнеров \texttt{Docker-Compose}~\cite{docker-compose}. С помощью \texttt{Docker}, можно создать изолированную среду для программного обеспечения, которое можно будет развернуть на различных операционных система без дополнительного вмешательства для обеспечения совместимости.

Тестирование производилось с помощью стандартных средств языка golang~---~go~test, позволяющих проводить как модульные, так и интеграционные тесты.

\section{Детали реализации}

Скрипт создания базы данных, включая создание индексов, органичений, триггеров и описанных выше ролей, приведена в Листингах ~\ref{lst:migration-1}~---~\ref{lst:migr5-1}.

Проектируемое приложение должно реализовывать доступ к разрабатываемой базе данных. Для этого может быть реализован и использован \texttt{REST API}~\cite{rest}, т.е. приложение будет являться web-сервером, API для взаимодействия с которым предстоит разработать.

При этом в разрабатываемом API необходимо учесть различные роли пользователей и соответствующий им функционал, описанный диаграммой вариантов использования и представленный на Рисунке~\ref{img:use-case}, и ограничения.

\texttt{REST API} проектируемого приложения представлен в Таблице~\ref{tbl:rest}.

\begin{landscape}
\begin{longtable}{|p{0.4\textwidth}|p{0.125\textwidth}|p{0.9\textwidth}|}
    \caption[Описание \texttt{REST API} реализуемого приложения]{Описание \texttt{REST API} реализуемого приложения} \label{tbl:rest}\\

    \hline
        Путь & Метод & Описание \\
    \endfirsthead

    \multicolumn{3}{l}
    {{\tablename\ \thetable{} -- продолжение}} \\\hline 
        Путь & Метод & Описание \\
    \endhead
    
    \multicolumn{3}{|r|}{{Продолжение на следующей странице}} \\ \hline
    \endfoot
    
    \hline \multicolumn{3}{|r|}{{Конец таблицы}} \\ \hline
    \endlastfoot
    
    \hline
    /api/v1/registration         & POST & Метод для регистрации нового пользователя в системе \\\hline
    /api/v1/login                & POST & Метод для авторизации существующего пользователя в системе \\\hline
    /api/v1/refresh              & GET & Метод обновления токена доступа\\\hline
    /api/v1/logout               & GET & Метод выхода из системы\\\hline
    /api/v1/users                & GET & Метод получения информации о пользователях системы \\\hline
    /api/v1/models               & POST GET PATCH DELETE & Методы взаимодействия с информацией о конфигурациях нейронных сетей \\\hline
    /api/v1/models/weights       & POST GET DELETE PATCH & Методы взаимодействия с информацией о конфигурациях весов нейронных сетей\\\hline
    /api/v1/admin/login          & POST & Метод авторизации администратора разработанного приложения \\\hline
    /api/v1/admin/refresh        & GET & Метод обновления токена доступа для администратора \\\hline
    /api/v1/admin/logout         & GET & Метод выхода из системы для администратора\\\hline
    /api/v1/admin/users          & GET DELETE & Методы управления информацией о пользователях для администратора\\\hline
    /api/v1/admin/users/blocked  & GET DELETE PATCH & Методы взаимодействия с информацией о блокировке пользователей \\\hline
    /api/v1/admin/models         & GET DELETE & Методы взаимодействия с информацией о конфигурациях нейронных сетей любого пользователя \\\hline
    /api/v1/admin/models/weights & GET DELETE & Методы взаимодействия с информацией о конфигурациях весов нейронных сетей любого пользователя \\\hline
    /api/v1/stat/login           & POST& Метод авторизации аналитика в системе \\\hline
    /api/v1/stat/refresh         & GET & Метод обновления токена доступа для аналитика\\\hline
    /api/v1/stat/logout          & GET & Метод выхода из ситсемы для аналитика \\\hline
    /api/v1/stat/users           & GET & Метод получения статистики по пользователям \\\hline
    /api/v1/stat/models          & GET & Метод получения статистики по конфигурациям нейронных сетей\\\hline
    /api/v1/stat/weights         & GET & Метод получения статистики по конфигурациям весов нейронных сетей\\
\end{longtable}
\end{landscape}

Развертывание приложения выполнялось посредством Docker-контейнеров и системой Docker-compose, файлы конфигурации для которых приведены в Листингах~\ref{lst:db-docker}--\ref{lst:prod-5}.

\section*{Вывод}

В данном разделе был обоснован выбор конкретных СУБД для решения задачи, описана общая архитектура реализованного программного продукта. Кроме того, было приведено описание архитектуры реализуемого приложения: описан архитектурный паттерн, использованный при проектировании, описаны конкретные компоненты приложения, а так же - определены средства реализации приложения, описан интерфейс взаимодействия с приложением, приведены детали реализации разрабатываемого приложения.