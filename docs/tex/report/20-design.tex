\chapter{Конструкторская часть}

В данном разделе будут спроектированы и приведены отношения таких сущностей разрабатываемой системы, как база данных конфигураций нейронных сетей и база данных кэширования, описан процесс репликации данных. Так же, будет проведено проектирование базы данных для хранения конфигурации нейронной сети на основе реляционной базы данных и проектирование базы данных для реализации кэширования данных на основе NoSQL in-memory базы данных.

\section{Проектирование отношений сущностей}

Так как в проектируемой базе данных наибольший объем будет занимать информация о конфигурации нейронной сети, т.е. о ее структуре и связанными с ней конфигурациями весов, то сохранять в кэшировать следует именно эту информацию --- это позволит добиться наиболее заметного снижения времени ответа разрабатываемого приложения~\cite{cache-what-is-it}. 

На Рисунке~\ref{img:replication} представлена схема сущностей, необходимых для реализации разрабатываемого приложения

\imgw{replication}{ht!}{\textwidth}{ER-модель разрабатываемой базы данных}

\section{Проектирование базы данных конфигураций нейронных сетей}

Система хранения будет реализована с использованием реляционной базы данных, в связи с этим проектирование базы выполняется в виде таблиц и связей между ними.

В разрабатываемой базе данных будет таблиц 10 таблиц, общая er-модель базы данных представлена на Рисунке~\ref{img:er-db}.

\imgw{er-db}{ht!}{0.9\textwidth}{ER-модель разрабатываемой базы данных}

Рассмотрим отдельно значения полей каждой из таблиц:

Таблица \texttt{migrations} содержит всего одно поле:
\begin{table}[!ht]
    \caption{Описание полей таблицы \texttt{migrations}}
    \label{tbl:migrations}
    \begin{center}
        \begin{tabular}{|p{0.2\textwidth}|p{0.7\textwidth}|}
            \hline
            \textbf{Поле} & \textbf{Значение} \\\hline
            \texttt{id} & Уникальный идентификатор завершенной миграции базы данных \\\hline
        \end{tabular}
    \end{center}
\end{table}

Данная таблица используется для фиксаций изменения структуры базы данных --- фиксации завершенных миграций базы данных~\cite{database-migrations}.

В таблице \texttt{users\_info} хранится информация о пользователях системы, а именно:

\begin{table}[!ht]
    \caption{Описание полей таблицы \texttt{users\_info}}
    \label{tbl:users}
    \begin{center}
        \begin{tabular}{|p{0.2\textwidth}|p{0.7\textwidth}|}
            \hline
            \textbf{Поле} & \textbf{Значение} \\\hline
            \texttt{id} & Уникальный идентификатор пользователя в системе. Этот идентификатор используется для связи владельца конфигурации нейронной сети с, собственно, конфигурации, загруженной в систему --- удалить данные из системы может только пользователь, загрузивший их \\\hline
            \texttt{username} & Имя пользователя для отображения в системе\\\hline
            \texttt{email} & Адрес электронной почты пользователя\\\hline
            \texttt{fullname} & Полное имя пользователя \\\hline
            \texttt{password\_hash} & Хеш пароля пользователя \\\hline
            \texttt{flags} & Значения этого поля определяют допустимые действия в системе\\\hline
            \texttt{blocked} & Время завершения блокировки пользователя --- до этого момента доступ пользователя к системе ограничен \\\hline
            \texttt{created\_at} & Время регистрации пользователя\\\hline
            \texttt{updated\_at} & Время последнего редактирования информации о пользователе \\\hline
        \end{tabular}
    \end{center}
\end{table}

В Таблице~\texttt{models} представлена информация о загруженных нейронных сетях, а именно:

\begin{table}[!ht]
    \caption{Описание полей таблицы \texttt{models}}
    \label{tbl:models}
    \begin{center}
        \begin{tabular}{|p{0.2\textwidth}|p{0.7\textwidth}|}
            \hline
            \textbf{Поле} & \textbf{Значение} \\\hline
            \texttt{id} & Уникальный идентификатор конфигурации \\\hline
            \texttt{title} & Имя пользователя для отображения в системе\\\hline
            \texttt{created\_at} & Время регистрации пользователя\\\hline
            \texttt{updated\_at} & Время последнего редактирования информации о пользователе \\\hline
            \texttt{owner\_id} & Идентификатор пользователя, загрузившего данную конфигурацию \\\hline
        \end{tabular}
    \end{center}
\end{table}

Такой способ представления нейронной сети позволяет получить всю доступную информацию о ней, зная лишь ее уникальный идентификатор.

Данная таблица имеет отношение многие-к-одному с таблицей \texttt{users\_info} --- у одного пользователя может иметься множество конфигураций нейронных сетей.

В таблице \texttt{structures} представлена информация о стуктуре нейронной сети, а именно:

\begin{table}[!ht]
    \caption{Описание полей таблицы \texttt{structures}}
    \label{tbl:structures}
    \begin{center}
        \begin{tabular}{|p{0.3\textwidth}|p{0.6\textwidth}|}
            \hline
            \textbf{Поле} & \textbf{Значение} \\\hline
            \texttt{id} & Уникальный идентификатор структуры \\\hline
            \texttt{title} & Название структуры нейронной сети \\\hline
            \texttt{model\_id} & Идентификатор модели, структура которой описывается данной записью в таблице \\\hline
        \end{tabular}
    \end{center}
\end{table}


В таблице \texttt{layers} представлена информация о слоях нейронной сети, а именно:

\begin{table}[!ht]
    \caption{Описание полей таблицы \texttt{layers}}
    \label{tbl:layers}
    \begin{center}
        \begin{tabular}{|p{0.3\textwidth}|p{0.6\textwidth}|}
            \hline
            \textbf{Поле} & \textbf{Значение} \\\hline
            \texttt{id} & Уникальный идентификатор слоя \\\hline
            \texttt{layer\_id} & Идентификатор слоя, полученный из входных данных \\\hline
            \texttt{limit\_func} & Идентификатор функции передачи (Конкретная функция в базе не хранится из-за проблем совместимости с различными реализациями нейронных сетей) \\\hline
            \texttt{activation\_func} & Идентификатор функции активации (Конкретная функция в базе не хранится из-за проблем совместимости с различными реализациями нейронных сетей) \\\hline
            \texttt{structure\_id} & Идентификатор структуры, которая включает в себя слой, описываемый данной записью в таблице \\\hline
        \end{tabular}
    \end{center}
\end{table}

В таблице \texttt{neurons} представлена информация о искуственных нейронах, содержащихся в нейронной сети, а именно:

\newpage

\begin{table}[!ht]
    \caption{Описание полей таблицы \texttt{neurons}}
    \label{tbl:neurons}
    \begin{center}
        \begin{tabular}{|p{0.2\textwidth}|p{0.7\textwidth}|}
            \hline
            \textbf{Поле} & \textbf{Значение} \\\hline
            \texttt{id} & Уникальный идентификатор нейрона \\\hline
             \texttt{neuron\_id} & Идентификатор нейрона, полученный из входных данных \\\hline
            \texttt{layer\_id} & Идентификатор слоя нейронной сети, в котором содержится искусственный нейрон, описываемый данной записью в таблице \\\hline
        \end{tabular}
    \end{center}
\end{table}


Таблица \texttt{neuronlinks} описывает связи между искуственными нейронами:

\begin{table}[!ht]
    \caption{Описание полей таблицы \texttt{neuron\_links}}
    \label{tbl:neuronlinks}
    \begin{center}
        \begin{tabular}{|p{0.2\textwidth}|p{0.7\textwidth}|}
            \hline
            \textbf{Поле} & \textbf{Значение} \\\hline
            \texttt{id} & Уникальный идентификатор связи \\\hline
            \texttt{link\_id} & Идентификатор связи, полученный из входных данных \\\hline
            \texttt{from\_id} & Идентификатор нейрона, в котором начинается данная связь \\\hline
            \texttt{to\_id} & Идентификатор нейрона, в котором завершается данная связь \\\hline
        \end{tabular}
    \end{center}
\end{table}

Далее рассматривается описание конфигурации весов нейронной сети.

Таблица \texttt{weightsinfo} описывает конфигурацию весов нейронной сети, а именно:

\begin{table}[!ht]
    \caption{Описание полей таблицы \texttt{weights\_info}}
    \label{tbl:weightsinfo}
    \begin{center}
        \begin{tabular}{|p{0.2\textwidth}|p{0.7\textwidth}|}
            \hline
            \textbf{Поле} & \textbf{Значение} \\\hline
            \texttt{id} & Уникальный идентификатор конфигурации весов нейронной сети \\\hline
            \texttt{name} & Название данной конфигурации --- используется пользователями для идентификации конфигураций \\\hline
            \texttt{created\_at} & Время добавления  данной конфигурации весов\\\hline
            \texttt{updated\_at} & Время последнего редактирования информации о данной конфигурации весов \\\hline
            \texttt{structure\_id} & Идентификатор структуры нейронной сети, которая описывается данной конфигурацией весов\\\hline
        \end{tabular}
    \end{center}
\end{table}

Таблица \texttt{neuron\_offsets} описывает вес смещения конкретных нейронов и содержит следующую информацию:

\begin{table}[!ht]
    \label{tbl:neuronoff}
    \caption{Описание полей таблицы \texttt{neuron\_offsets}}
    \begin{center}
        \begin{tabular}{|p{0.3\textwidth}|p{0.6\textwidth}|}
            \hline
            \textbf{Поле} & \textbf{Значение} \\\hline
            \texttt{id} & Уникальный идентификатор веса смещения нейрона \\\hline
            \texttt{value} & Величина смещения \\\hline
            \texttt{neuron\_id} & Идентификатор нейрона, смещение которого описывает данная запись в таблице \\\hline
            \texttt{weights\_info\_id} & Идентификатор конфигурации весов нейронной сети, в которую входит данный вес смещения\\\hline
        \end{tabular}
    \end{center}
\end{table}

Таблица \texttt{link\_weights} описывает вес конкретных связей между нейронами и содержит следующую информацию:

\begin{table}[!ht]
    \label{tbl:neuronweights}
    \caption{Описание полей таблицы  \texttt{link\_weights}}
    \begin{center}
        \begin{tabular}{|p{0.3\textwidth}|p{0.6\textwidth}|}
            \hline
            \textbf{Поле} & \textbf{Значение} \\\hline
            \texttt{id} & Уникальный идентификатор веса связи между нейронами \\\hline
            \texttt{value} & Вес связи \\\hline
            \texttt{link\_id} & Идентификатор связи, вес которой описывает данная запись в таблице \\\hline
            \texttt{weights\_info\_id} & Идентификатор конфигурации весов нейронной сети, в которую входит данный вес смещения\\\hline
        \end{tabular}
    \end{center}
\end{table}

\section{Проектирование базы данных кэширования}

База данных кеширования будет реализована посредством NoSQL in-memory базы данных. Как указывалось ранее, способ представления данных в таких базах данных зависит от выбранной базы, в связи с чем проектирование такой базы ограничивается высокоуровневым описанием хранящихся данных.

Ранее указывалось, что разрабатываемая база данных кэширования должна предоставлять возможность хранения информации о нейронной сети и конфигурации весов нейронной сети.

Для информации о конфигурации нейронной сети это можно представить образом, представленным в Таблице \ref{tbl:memcache}

\begin{table}[!ht]
    \caption{Описание хранилища \texttt{model}}
    \begin{center}
        \label{tbl:memcache}
        \begin{tabular}{|p{0.3\textwidth}|p{0.6\textwidth}|}
            \hline
            \textbf{Поле} & \textbf{Значение} \\\hline
            \texttt{id} & Уникальный идентификатор нейронной сети \\\hline
            \texttt{info} & Информация о конфигурации нейронной сети \\\hline
        \end{tabular}
    \end{center}
\end{table}

Для информации о конфигурации весов представление аналогично и представлено в Таблице~\ref{tbl:memcachew}:

\begin{table}[!ht]
    \caption{Описание хранилища \texttt{weight}}
    \begin{center}
        \label{tbl:memcachew}
        \begin{tabular}{|p{0.3\textwidth}|p{0.6\textwidth}|}
            \hline
            \textbf{Поле} & \textbf{Значение} \\\hline
            \texttt{id} & Уникальный идентификатор нейронной сети \\\hline
            \texttt{info} & Информация о конфигурации весов нейронной сети \\\hline
        \end{tabular}
    \end{center}
\end{table}

Конкретная реализация данных хранилищ будет определена после выбора NoSQL базы данных для реализации хранилища кэша, однако уже на текущем этапе можно определить следующие требования к итоговой реализации данных хранилищ:
\begin{itemize}
    \item возможность сохранения информации "как есть"{} --- т.е. отсутствие необходимости в специальной подготовке данных перед сохранением в хранилище;
    \item хранение информации о конфигурациях в виде, в котором эта информация будет передана пользователю --- т.е. информация, хранящаяся в кэше не должна требовать дополнительной обработки после получения ее из хранилища.
\end{itemize}

\section{Соблюдение целостности данных}

Для соблюдения целостности данных в проектируемой базе данных будут использованы внешние ключи, ограничения, триггеры, а так же --- ролевая модель.\\

\noindent\textbf{Триггеры}\\

На Рисунке~\ref{img:upd_model} представлена схема алгоритма триггера,  срабатывающего после обновления строки в таблице \texttt{models}.

\imgw{upd_model}{ht!}{0.22\textwidth}{Схема алгоритма триггера обновления информации о нейронной сети}

На Рисунке~\ref{img:upd_weight} представлена схема алгоритма триггера,  срабатывающего после обновления строки в таблице \texttt{weights\_info}.

\imgw{upd_weight}{ht!}{0.22\textwidth}{Схема алгоритма триггера обновления информации о конфигурации весов нейронной сети}

Так же, в проектируемой базе данных должны быть реализованы триггеры, осуществляющие синхронизацию данных между, собственно, базой данных и кэшем. 

На Рисунке~\ref{img:delete_model_tnt} представлена схема алгоритма триггера, удаляющего информацитю о конфигурации нейронной сети из кэша.

\imgw{delete_model_tnt}{ht!}{0.3\textwidth}{Схема алгоритма триггера удаления информации о нейронной сети из кэша}

Данный тригер должен срабатывать при обновлении или удалении записей из следующих таблиц: \texttt{models}, \texttt{structures}, \texttt{weights\_info}.

\newpage

На Рисунке~\ref{img:delete_weight_tnt} представлена схема алгоритма триггера, удаляющего информацитю о конфигурации весов нейронной сети из кэша.

\imgw{delete_weight_tnt}{ht!}{0.3\textwidth}{Схема алгоритма триггера удаления информации о конфигурации весов нейронной сети из кэша}

Данный тригер должен срабатывать при обновлении или удалении записей из следующих таблиц: \texttt{models}, \texttt{structures}, \texttt{weights\_info}.\\

\noindent\textbf{Внешние ключи}\\

В проектируемой базе данных используются следующие внешние ключи:
\begin{itemize}
	\item для таблицы \texttt{models} -- внешний ключ \texttt{owner\_id}, ссылающийся на поле~\texttt{id} таблицы~\texttt{users\_info};
	\item для таблицы \texttt{structures} -- внешний ключ \texttt{model\_id}, ссылающийся на поле~\texttt{id} таблицы~\texttt{models};
		\item для таблицы \texttt{layers} -- внешний ключ, \texttt{strucuure\_id}, ссылающийся на поле~\texttt{id} таблицы~\texttt{structures};
		\item для таблицы \texttt{neurons} -- внешний ключ, \texttt{layer\_id}, ссылающийся на поле~\texttt{id} таблицы~\texttt{layers};
		\item для таблицы \texttt{neuron\_links}:
			\begin{itemize}
				\item внешний ключ \texttt{from\_id}, ссылающийся на поле~\texttt{id} таблицы~\texttt{neurons};
				\item  внешний ключ \texttt{to\_id}, ссылающийся на поле~\texttt{id} таблицы~\texttt{neurons};
			\end{itemize}
		\item для таблицы \texttt{weights\_info} -- внешний ключ, \texttt{structure\_id}, ссылающийся на поле~\texttt{id} таблицы~\texttt{structures};
		\item для таблицы \texttt{neuron\_offsets}:
			\begin{itemize}
				\item внешний ключ \texttt{weights\_info\_id}, ссылающийся на поле \texttt{id} таблицы \texttt{weights\_info};
				\item внешний ключ \texttt{neuron\_id}, ссылающийся на поле~\texttt{id} таблицы~\texttt{neurons};
			\end{itemize}
		\item для таблицы \texttt{link\_weights}:
			\begin{itemize}
				\item внешний ключ \texttt{weights\_info\_id}, ссылающийся на поле~\texttt{id} таблицы~\texttt{weights\_info};
				\item внешний ключ \texttt{link\_id}, ссылающийся на поле~\texttt{id} таблицы~\texttt{neuron\_links};
			\end{itemize}
\end{itemize}

\noindent\textbf{Ограничения}\\

В проектируемой базе данных будет использоваться одно ограничение: поле \texttt{flags} таблицы \texttt{user\_info} должно быть неотрицательным.\\

\noindent\textbf{Ролевая модель}\\

Ролевая модель базы данных должна содержать 3 роли:
\begin{itemize}
	\item обычный пользователь --- имеет возможность:
		\begin{itemize}
			\item добавлять и получать записи из таблицы~\texttt{users\_info};
			\item добавлять, обновлять, получать и удалять записи из таблицы~\texttt{models};
		\end{itemize}
	\item аналитик --- имеет возможность получать записи из таблиц \texttt{users\_info}, \texttt{models}, \texttt{weights\_info};
	\item администратор --- имеет возможность добавлять, обновлять, получать и удалять записи из любых таблиц.
\end{itemize}

\section*{Вывод}

В данном разделе были спроектированы и приведены отношения таких сущностей разрабатываемой системы, как база данных конфигураций нейронных сетей и база данных кэширования, описан процесс репликации данных. Так же, было проведено проектирование базы данных для хранения конфигурации нейронной сети на основе реляционной базы данных и проектирование базы данных для реализации кэширования данных на основе NoSQL in-memory базы данных.