\chapter{Аналитическая часть}

В данном разделе проведена формализация задачи: рассмотрено понятие нейронной сети, описан способ представления искуственного нейрона, а так же структура конфигурации нейронной сети. Так же, проведена формализация данных, приведёно сравнение существующих способов хранения данных. Рассмотрены системы управления базами данных, оптимальные для решения поставленной задачи. Описаны проблемы кэшированния данных и предложены методы их решения.

\section{Формализация задачи}

Нейронные сети являются подмножеством машинного обучения и лежат в основе алгоритмов глубокого обучения. Их название и структура вдохновлены человеческим мозгом, а алгоритм работы основывается на способе, которым биологические нейроны передают сигналы друг другу.~\cite{neural}

Простейшая нейронная сеть включает в себя входной слой, выходной (или целевой) слой и, между ними, скрытый слой. Слои соединены через узлы (искусственные нейроны). Эти соединения образуют <<сеть>> -- нейронную сеть -- из взаимосвязанных узлов.~\cite{sas-neural} Пример приведен на Рисунке~\ref{img:network}.

\imgw{network}{h!}{0.7\textwidth}{Пример схемы нейронной сети}

В общем случае искусственный нейрон можно образом, приведенном на Рисунке~\ref{img:neuron}.

\imgw{neuron}{h!}{0.7\textwidth}{Общая схема искусственного нейрона}

Данное представление применимо к любому виду нейронных сетей -- вне зависимости от типа, нейронные сети реализуются путем упорядочивания нейронов в слои и последующим связыванием соответствующих слоев между собой. \cite{sharkawy-neural}

В процессе обучения нейронной сети используется так называемая обучающая выборка -- заранее подготовленный набор данных, отражающий суть рассматриваемой предметной области \cite{sharkawy-neural}. В зависимости от содержимого обучающей выборки результирующие весовые конфигурации нейронной сети (т.~е. веса связей между нейронами, а так же смещения отдельно взятых нейронов) могут отличаться \cite{mit-neural}. В связи чем этим одна и та же структура нейронной сети может переиспользована для работы с различными предметными областями.

\section{Варианты использования}

В рамках данной курсовой работы выделены следующие роли пользователей системы:
\begin{itemize}
    \item пользователь --- роль, позволяющая загружать, получать, изменять и удалять существующие конфигурации нейронных сетей;
    \item аналитик --- роль, позволяющая получать статистику работы системы -- количество загрузок, изменений конфигураций нейронных сетей и т.~д.;
    \item администратор --- роль, позволяющая изменять, удалять любую конфигурацию нейронной сети, блокировать других пользователей, удалять существующих пользователей.
\end{itemize}

Варианты использования, предусмотренные в рамках данной курсовой работы приведены на Рисунке~\ref{img:use-case}:

\imgw{use-case}{h!}{0.8\textwidth}{Диаграмма вариантов использования}

\section{Формализация данных}

Конфигурация нейронной сети включает в себя информацию о~\cite{sharkawy-neural}:

\begin{itemize}
    \item отдельных слоях:
        \begin{itemize}
            \item функции активации конкретных слоев;
            \item функции передачи конкретных слоев;
        \end{itemize}
    \item отдельных нейронах:
        \begin{itemize}
            \item связях между нейронами;
            \item принадлежности конкретных нейронов тому или иному слою;
        \end{itemize}
    \item результирующих значениях весов:
        \begin{itemize}
            \item смещения конкретных нейронов;
            \item веса связей между конкретными нейронами.
        \end{itemize}
\end{itemize}

Отдельно стоит отметить, что:
\begin{itemize}
    \item структура нейронной сети представляет собой, в общем случае, ориентированный граф с $n$-входами и $m$-выходами;
    \item обучающие выборки не являются обязательным компонентом конфигурации нейронной сети.
\end{itemize}

Результирующая ER-модель представлена на Рисунке~\ref{img:er}.

\imgw{er}{h!}{\textwidth}{er-модель}

Таким образом, система хранения конфигураций нейронных систем должна обеспечивать возможность хранения всей приведенной информации.

\newpage

\section{Базы данных и системы управления базами данных}

База данных -- это упорядоченный набор структурированной информации или данных, которые обычно хранятся в электронном виде в компьютерной системе~\cite{database}. Для управления базами данных используется системы управления базами данных (далее СУБД). 

СУБД -- это комплекс программно-языковых средств, позволяющих создать базы данных и управлять данными. Иными словами, СУБД — это набор программ, позволяющий организовывать, контролировать и администрировать базы данных.~\cite{subd}

\subsection{Классификация баз данных по способу хранения}

Существует множество типов баз данных \cite{db-comparison}:
\begin{itemize}
    \item дореляционные:
    \begin{itemize}
        \item плоские;
        \item иерархические;
        \item сетевые;
    \end{itemize}
    \item реляционные;
    \item постреляционные:
    \begin{itemize}
        \item \texttt{NoSQL}:
            \begin{itemize}
                \item «ключ-значение»;
                \item документо-ориентированные;
                \item графовые;
                \item колоночные;
                \item временных рядов;
            \end{itemize}
        \item NewSQL
    \end{itemize}
\end{itemize}

В настоящее время наиболее популярными являются реляционные базы данных, а так же различные \texttt{NoSQL} решения~\cite{modern-db-comparison}.

Так как любая нейронная сеть может быть представлена описанным выше способом, для хранения конфигурации нейронной сети целесообразно использовать реляционные или графовые базы данных.\\

\noindent\textbf{Реляционные базы данных}\\

Реляционные базы данных, появившиеся в 1970-х, являются наиболее распространенным типом баз данных. Так, в настоящее время 7 из 10 наиболее популярных баз данных являются реляционными~\cite{db-ranking}. Кроме того, в реляционных базах используется стандартизированные язык описания и управления данными - \texttt{SQL}

В реляционных базах данные представлены в виде строк в таблицах и связей между конкретными строками, в связи с чем такие базы данных подходят для хранения строго структурированной и типизированной информации, причем их можно разделить на 2 группы:
\begin{itemize}
    \item строковые -- используются в системах, в которых предполагается больше количество операций вставки и обновления данных, использующих транзакции (англ. OLTP~\cite{oltp}).
    \item колоночные -- используются в системах, в которых предполагается большое количество операций выборки данных, в том числе посредством сложных запросов со множественными объединениями таблиц, для анализа хранящейся информации (англ. OLAP~\cite{olap}).
\end{itemize}

Реляционные базы данных предоставляют механизм транзакций и удовлетворяют принципам \texttt{ACID}~\cite{acid}, в связи с чем, несмотря на сложность внесения изменений в уже созданную базу данных, получили широкое распространение во многих сферах.~\cite{modern-db-comparison}\\

\noindent\textbf{Графовые базы данных}\\

Графовые базы данных --- это техническая реализация теории графов, концепции, которая была введена в прикладную математику около 200 лет назад~\cite{graphdatabases}. 

Такие базы данных представляют данные в виде узлов графа и связей между ними, уделяя особое внимание связям между данными. Благодаря своей структуре, не требуют множественных объединений таблиц, свойственных реляционным базам данных и предоставляют гораздо более гибкую модель для хранения данных \cite{graphdatabases}.

В настоящее время существуют 2 основных типа графовых баз данных, основное отличие которых заключается в использующихся моделях данных, которые описывают:
\begin{itemize}
    \item ресурсы (англ. Resource Descriptive Framework), далее~\texttt{RDF};
    \item свойства ресурсов (англ. Property Graph Databases), далее~\texttt{PGD}.
\end{itemize}

\texttt{RDF}-базы фокусируются на связях между узлами графа. В таких базах данных узлы не содержат никакой информации, а хранимые данные сосредоточены в связях между ними. Такие базы данных наиболее эффективны для публикации данных и обмена информацией.~\cite{graph-vs-rel}

\texttt{PGD}-базы фокусируются на логической модели данных, пытаясь реализовать наиболее эффективное по времени поиска и объему, требующейся для хранения памяти, представление данных. \cite{graph-vs-rel}

Однако, что свойственно нереляционным базам данных, \texttt{RDF} и \texttt{PGD}, в большинстве случаев, не обладают механизмом транзакций, не удовлетворяют принципам \texttt{ACID}, а так же не имеют стандартного интерфейса для описания и управления данными.~\cite{rel-vs-nosql}

\subsection{Выбор типа базы данных для решения задачи}

База данных для решения задачи должна обладать следующими свойствами:
\begin{itemize}
    \item наличие строго типизированного языка описания структуры данных и связей между ними (англ.~DDL);
    \item наличие строго типизированного языка управления данными и связями между ними (англ.~DML);
    \item поддержка принципов \texttt{ACID}, транзакций;
\end{itemize}

\begin{table}[!ht]
    \caption{Сравнение типов баз данных}
    \label{tbl:comparison}
    \begin{center}
        \begin{tabular}{|c|c|c|}
            \hline
            \textbf{Критерий} & \textbf{Реляционные} & \textbf{Графовые} \\\hline
            DDL & + & --- \\\hline
            DML & + & $\pm$ \\\hline
            ACID & + & $\pm$ \\\hline
        \end{tabular}
    \end{center}
\end{table}

В связи с вышеуказанными особенностями конфигураций нейронных сетей, а так же результатами сравнения, приведенными в Таблице~\ref{tbl:comparison}, наиболее предпочтительными являются реляционные базы данных.

Так как разрабатываемая система предполагает большое количество транзакций с операциями вставки, удаления и редактирования данных.

\section{Кэширование данных}

В современном мире все больше и больше людей подключаются к Интернету. На текущий момент более 65\% населения земного шара имеют подключение к Интернету \cite{internetusers}.

Учитывая все возрастающее число пользователей сети Интернет, для повышения быстродействия программного обеспечения, используется метод кэширования данных --- данные, которые, предположительно, не будут изменены в ближайшее время, сохраняются в независимое хранилище данных. В качестве такого хранилища, обычно, используют NoSQL~\cite{nosql} in-memory базы данных. Такие базы данных хранят данные в оперативной памяти, что повышает скорость доступа к данным за счет отсутствия затрат на чтение данных со вторичных носителей информации.

\subsection{Проблемы кэширования данных}

\noindent\textbf{Синхронизация данных}\\

Данную проблему можно решить двумя путями: 
\begin{itemize}
    \item реализации на уровне базы данных реализуемого программного продукта триггеров,  осуществляющих синхронизацию данных между базой данных и кэшем при изменении / удалении данных из базы;
    \item реализация на уровне разрабатываемого приложения функций, обеспечивающих синхронизацию данных между базой данных и кэшем при изменении / удалении данных из базы.
\end{itemize}

\noindent\textbf{Проблема <<холодного старта>>}\\

При первоначальном запуске системы кэш пуст, в связи с чем все запросы будут напрямую обрабатываться базой данных до тех пор, пока кэш не будет <<разогрет>> в достаточной мере, чтобы снизить нагрузку на базу данных.

У этой проблемы есть множество решений. Далее приведены два наиболее популярных:
\begin{itemize}
    \item использование базы данных с журналированием всех операций -- при перезагрузке можно восстановить предыдущее состояние кэша с помощью журнала событий, который хранится на диске.
    \item использование кэша с поддержкой <<снимков>> хранилища -- при перезагрузке можно будет сразу восстановить наиболее актуальное состояние кеша до перезагрузки с помощью <<снимка>>, хранящегося на диске~\cite{tarantool-snapshots}.
\end{itemize}

В обоих случаях может потребоваться синхронизация кэша и базы данных, так как за время перезагрузки кеша данные, находящиеся в нем могли перестать быть актуальными.

\section*{Вывод}

В данном разделе была проведена формализация задачи: рассмотрено понятие нейронной сети, описан способ представления искуственного нейрона, а так же структура конфигурации нейронной сети. Так же, была проведена формализация данных, приведёно сравнение существующих способов хранения данных. Рассмотрены системы управления базами данных, оптимальные для решения поставленной задачи. Описаны проблемы кэшированния данных и предложены методы их решения.