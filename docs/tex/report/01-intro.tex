\chapter*{ВВЕДЕНИЕ}
\addcontentsline{toc}{chapter}{ВВЕДЕНИЕ}

В настоящее время популярность алгоритмов машинного обучения продолжает неуклонно расти, в связи с чем все чаще возникает проблема хранения результатов работы специалистов в области машинного обучения, в частности - конфигураций нейронных сетей.~\cite{popularity}

Разработка системы хранения конфигураций нейронных позволит ускорить процесс работы специалистов: такая система гарантирует сохранность полученных ранее результатов работы, а главное --- позволяет централизованно взаимодействовать с результатами работы сразу нескольким людям.

Такая система наиболее актуальна для небольших команд разработчиков, использующих в своей деятельности нейронные сети, позволяет хранить конфигурацию любой нейронной сети, в связи с чем является универсальным решением.

Целью данной работы является проектирование и разработка базы данных для хранения конфигураций нейронных сетей.
    
Для достижения данной цели необходимо решить следующие задачи:

\begin{itemize}
    \item проанализировать варианты представления данных и выбрать подходящий вариант для решения задачи;
    \item проанализировать системы управления базами данных и выбрать подходящую систему для хранения данных.
    \item спроектировать базу данных, описать ее сущности и связи;
    \item реализовать интерфейс для доступа к базе данных;
    \item реализовать программное обеспечение, позволяющее взаимодействовать со спроектированной базой данных.
\end{itemize}