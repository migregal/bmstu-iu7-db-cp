\chapter{Исследовательская часть}

В данном разделе представлена постановка эксперимента по сравнению занимаемого времени для получения данных о конфигурации нейронных сетей посредством реализованной базы данных и приложения, обеспечивающего доступ к ней с использованием и без использования кэширования:  описан эксперимент, приведены технические характеристики устройств, использованных во время тестирования, приведены результаты измерений, полученные в ходе эксперимента.

\section{Цель эксперимента}
% \noindent\textbf{Цель эксперимента}\\

Целью эксперимента является сравнение времени, требуемого для получения данных о конфигурации нейронной сети посредством разработанного приложения с и без использования кэширования данных.

Для достижения этой цели требуется:
\begin{itemize}
    \item сгенерировать конфигурации нейронных сетей отвечающие требуемым характеристикам'
    \item загрузить сгенерированные конфигурации в разработанную базу данных;
    \item произвести экспериментальные замеры времени, необходимые для получения информации о сгенерированных конфигурациях нейронных сетей.
\end{itemize}

\section{Описание эксперимента}
% \noindent\textbf{Описание эксперимента}\\

Сравнить занимаемое время можно при помощи отключения реализованного механизма кэширования. Для этого будет достаточно отключить базу данных, хранящую данные о кэшировании и каждый раз выполнять запрос напрямую к базе данных хранящую информацию о конфигурациях нейронных сетей.

Для проведения будут использоваться разные размеры запрашиваемых конфигураций нейронных сетей. Будут произведены операции, для того чтобы запрашиваемые конфигурации нейронных сетей могли оказаться в кэше.

В поставленном эксперименте одна нейронная сеть будет состоять из 8, 16, 32 \dots, 256 слоев по 16, 32, 64, 128 нейронов в каждом. Связи между нейронами организованы следующим образом: каждый нейрон из $j$-ого слоя соединяется с каждым нейроном из $j+1$-ого слоя. 

Таким образом, будут рассмотрены конфигурации нейронных сетей с количеством связей между нейронами от $1792$, до $4177920$ штук.

Пример такой нейронной сети приведен на Рисунке \ref{img:organization}.

\imgw{organization}{ht!}{0.8\textwidth}{Общая структура использующейся конфигурации нейронной сети}

\section{Технические характеристики}
% \noindent\textbf{Технические характеристики}\\

Эксперимент производился с использованием двух ЭВМ - сервера  и клиента.

В качестве сервера выступал \texttt{MacBook~Pro~2020}, технические характеристики:
\begin{itemize}
	\item процессор: Intel~Core™~i7-1068NG7~\cite{i7}~CPU~@~2.30ГГц;
	\item память: 16~Гб;
	\item операционная система: \texttt{macOS~Big~Sur}~\cite{bigsur} 11.6.
\end{itemize}


В качестве клиента использовался \texttt{HP ProBook 440 G5}, технические характеристики:
\begin{itemize}
	\item процессор: Intel~Core™~i5-8250U~\cite{i5}~CPU~@~1.60~Ггц;
	\item память: 32~Гб;
	\item операционная система: Manjaro~\cite{manjaro}~Linux~\cite{linux} 21.2.26~64-bit.
\end{itemize}

Исследование проводилось на ноутбуках, подключенным к сети электропитания. Во время тестирования ноутбуки были нагружены только встроенными приложениями окружения рабочего стола, окружением рабочего стола, а также непосредственно тестируемой системой.

\section{Результат эксперимента}
% \noindent\textbf{Результат эксперимента}\\

Полученные в ходе эксперимента данные, представлены в Таблице~\ref{tbl:time}.

\begin{table}[ht]
	\small
	\begin{center}
		\caption{Замеры времени для различных конфигураций, c}
		\label{tbl:time}
		\begin{tabular}{|c|c|c|c|c|c|c|c|c|}
			\hline
			\bfseries Количество & \multicolumn{8}{c|}{\bfseries Количество слоев, шт.} \\\cline{2-9}
			\bfseries нейронов в & \multicolumn{2}{c|}{\bfseries 16} & \multicolumn{2}{c|}{\bfseries 32} &
			\multicolumn{2}{c|}{\bfseries 64} & \multicolumn{2}{c|}{\bfseries 128}\\\cline{2-9}
			\bfseries слое, шт. & БД & Кэш & БД & Кэш & БД & Кэш & БД & Кэш
			\csvreader{inc/csv/res.csv}{}
			{\\\hline \csvcoli&\csvcolii&\csvcoliii&\csvcoliv&\csvcolv&\csvcolvi&\csvcolvii&\csvcolviii&\csvcolix}
			\\\hline
		\end{tabular}
	\end{center}
\end{table}

Для большей наглядности представим те же данные в виде графиков. 

На Рисунке~\ref{plt:16} представлено сравнение результатов замеров времени для конфигураций с 16~нейронами в каждом слое.

\begin{figure}[htp]
	\centering
	\begin{tikzpicture}
		\begin{axis}[
			axis lines=left,
			xlabel={Количество слоев в нейронной сети},
			ylabel={Время, с},
			legend pos=north west,
			ymajorgrids=true,
			xtick={32,64,128,256}
		]
			\addplot table[x=size,y=db,col sep=comma] {inc/csv/res16.csv};
			\addplot table[x=size,y=cache,col sep=comma] {inc/csv/res16.csv};
            \legend{Без кэширования, С кэшированием}
		\end{axis}
	\end{tikzpicture}
	\captionsetup{justification=centering}
	\caption{Зависимость времени от размера конфигурации нейронной сети (16 нейронов в слое)}
	\label{plt:16}
\end{figure}

На Рисунке~\ref{plt:32} представлено сравнение результатов замеров времени для конфигураций с 32~нейронами в каждом слое.

\begin{figure}[htp]
	\centering
	\begin{tikzpicture}
		\begin{axis}[
			axis lines=left,
			xlabel={Количество слоев в нейронной сети},
			ylabel={Время, с},
			legend pos=north west,
			ymajorgrids=true,
			xtick={32,64,128,256}
		]
			\addplot table[x=size,y=db,col sep=comma] {inc/csv/res32.csv};
			\addplot table[x=size,y=cache,col sep=comma] {inc/csv/res32.csv};
            \legend{Без кэширования, С кэшированием}
		\end{axis}
	\end{tikzpicture}
	\captionsetup{justification=centering}
	\caption{Зависимость времени от размера конфигурации нейронной сети (32 нейрона в слое)}
	\label{plt:32}
\end{figure}

На Рисунке~\ref{plt:64} представлено сравнение результатов замеров времени для конфигураций с 64~нейронами в каждом слое.

\begin{figure}[htp]
	\centering
	\begin{tikzpicture}
		\begin{axis}[
			axis lines=left,
			xlabel={Количество слоев в нейронной сети},
			ylabel={Время, с},
			legend pos=north west,
			ymajorgrids=true,
			xtick={32,64,128,256}
		]
			\addplot table[x=size,y=db,col sep=comma] {inc/csv/res64.csv};
			\addplot table[x=size,y=cache,col sep=comma] {inc/csv/res64.csv};
            \legend{Без кэширования, С кэшированием}
		\end{axis}
	\end{tikzpicture}
	\captionsetup{justification=centering}
	\caption{Зависимость времени от размера конфигурации нейронной сети (64 нейрона в слое)}
	\label{plt:64}
\end{figure}

\newpage

На Рисунке~\ref{plt:128} представлено сравнение результатов замеров времени для конфигураций с 128~нейронами в каждом слое.

\begin{figure}[htp]
	\centering
	\begin{tikzpicture}
		\begin{axis}[
			axis lines=left,
			xlabel={Количество слоев в нейронной сети},
			ylabel={Время, с},
			legend pos=north west,
			ymajorgrids=true,
			xmajorgrids=true,
			xtick={32,64,128,256}
		]
			\addplot table[x=size,y=db,col sep=comma] {inc/csv/res128.csv};
			\addplot table[x=size,y=cache,col sep=comma] {inc/csv/res128.csv};
            \legend{Без кэширования, С кэшированием}
		\end{axis}
	\end{tikzpicture}
	\captionsetup{justification=centering}
	\caption{Зависимость времени от размера конфигурации нейронной сети (128 нейронов в слое)}
	\label{plt:128}
\end{figure}

\section*{Вывод}

В результате анализа полученных экспериментально данных о времени, необходимом для получения данных о конфигурации нейронной сети, кэширование данных показало более высокие результаты:
\begin{itemize}
    \item вне зависимости от размера нейронной сети, приложение с кэшированием всегда оказывалось быстрее;
    \item при большом размере (256 слоев по 128 нейронов в каждом) нейронной сети, приожение с кэшированием оказалось в 39.4 раза быстрее (при условии, что запрашиваемые данные находятся в кэше).
\end{itemize}

Тем не менее, в ходе данного эксперимента не учитывался размер хранимой в кэше информации. Так, для конфигурации сети из 256~слоев по 128~нейронов в каждом, требуется выделить порядка 11~Мб памяти в кэше, при условии сжатия ее посредством \texttt{gzip}~\cite{gzip}.

В связи с этим можно сделать вывод, что в кэше следует хранить только большие конфигурации нейронных сетей (более 32 слоев), так как, при меньших размерах конфигурации, кэширование не обеспечивает сопоставимого прироста производительности по времени.