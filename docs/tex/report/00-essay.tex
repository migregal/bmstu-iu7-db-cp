\begin{essay}{}
    \noindent\textbf{Ключевые слова}: Базы Данных, SQL, NoSQL, Кэширование данных, Нейронные сети, Конфигурация нейронной сети\\
        
    Объектом разработки является базы данных для хранения конфигураций нейронных сетей.
    
    Цель работы --- спроектировать и разработать базу данных для хранения конфигураций нейронных сетей.
    
    Для достижения данной цели необходимо решить следующие задачи:
    
    \begin{itemize}
        \item проанализировать варианты представления данных и выбрать подходящий вариант для решения задачи;
        \item проанализировать системы управления базами данных и выбрать подходящую систему для хранения данных.
        \item спроектировать базу данных, описать ее сущности и связи;
        \item реализовать интерфейс для доступа к базе данных;
        \item реализовать программное обеспечение, позволяющее взаимодействовать со спроектированной базой данных.
    \end{itemize}
    
    В результате выполнения работы была спроектирована и разработана база данных для хранения конфигураций нейронных сетей.
    
    По результатам экспериментальных измерений, использование кэширования при получении информации из базы данных позволяет снизить времени отклика системы вплоть до 39 раз, при условии, что запрашиваемая информация находится в кэше.
\end{essay}