\chapter*{ЗАКЛЮЧЕНИЕ}
\addcontentsline{toc}{chapter}{ЗАКЛЮЧЕНИЕ} 

В результат выполненя данной курсовой работы была достигнута ее цель --- спроектирована и разработана база данных для хранения конфигураций нейронных сетей.
    
В процессе достижения данной цели достижения данной цели были решены следующие задачи:

\begin{itemize}
    \item проанализированы варианты представления данных и выбран подходящий вариант для решения задачи;
    \item проанализированы системы управления базами данных и выбрана подходящая система для хранения данных.
    \item спроектирована базу данных, описаны ее сущности и связи;
    \item реализован интерфейс для доступа к базе данных;
    \item реализовано программное обеспечение, позволяющее взаимодействовать со спроектированной базой данных.
\end{itemize}

В ходе курсовой работы были получены знания в области проектирования баз данных, кжширования данных, проектирования архитектуры web-приложений. Были изучены различные типы баз данных, а так же способы хранения данных, используемые в тех или иных базах данных.

В результате проделанной работы, было разработано программное обеспечение, позволившее снизить время отклика всей системы за счет внедрения механизма кэширования данных.

В ходе выполнения исследовательской части работы было установлено, что кэширование данных всегда приводит к росту производительности системы с точки зрения времени отклика, однако наиболее оптимальным оказался вариант с кэшированием сравнительно больших объемов данных --- в этом случае кэширование данных может иметь выигрыш по времени в 39~раз.